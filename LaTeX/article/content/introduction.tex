\section{Introdução}

% Estruture como um funil, começando do mais geral para o mais especifico
%
% Simulação e controle
% Redes Neurais
% PINNs
% PIRNNs
% Deixe os tanques e software-in-the-loop para a metodologia
%
A simulação e o controle de processos são fundamentais para a indústria, desempenhando um papel essencial na eficiência e segurança dos processos. Através da simulação, é possível prever o comportamento de sistemas complexos sem a necessidade de intervenções físicas, o que permite identificar problemas, testar soluções e otimizar processos antes comprometer recursos físicos. O controle, por sua vez, assegura que os sistemas operem nos parâmetros estabelecidos, ajustando variáveis em tempo real para manter a estabilidade e a produtividade.

O trabalho conjunto entre simulação e controle é particularmente bem exemplificado no controle preditivo baseado em modelo (\textit{Model Predictive Control} — MPC), uma abordagem avançada que tem sido amplamente aplicada em indústrias de processos, como químicas e petroquímicas, desde o final da década de 1970 \citep{garcia_1989}. Nesse contexto, o uso de um modelo preciso do sistema é fundamental, por permitir prever a resposta futura do processo com base em seu estado atual e nas possíveis ações de controle.

Entretanto, na prática, a indústria enfrenta uma série de desafios práticos para implementação de um sistema assim. Por exemplo, é preciso que a resposta da simulação seja rápida o suficiente para fornecer previsões em tempo real para o controlador tomar decisões eficazes a cada novo ciclo. Dependendo do modelo utilizado, isso exige alto poder computacional, o que pode ser um problema para sistemas embarcados.% Além disso, a instrumentação do sistema deve ser precisa e confiável, com sensores que operem sincronizadamente e forneçam dados continuamente e sem atrasos.

% Um exemplo de sistema que pode apresentar esses desafios é a simulação e controle de nível em tanques esféricos. A dinâmica do nível de líquido nesses tanques é não linear devido à variação da área da seção transversal com a altura do líquido, o que significa que pequenas variações na vazão de entrada podem resultar em grandes mudanças na altura do líquido \citep{priya_2012}.

Uma abordagem diferente para esses problemas é o uso de redes neurais artificiais (\textit{Artificial Neural Networks} — ANNs). Elas são consideradas aproximadoras de funções universais \citep{csaji_2001} e têm o potencial de serem mais rápidas que os métodos numéricos convencionais por terem uma arquitetura baseada em operações simples e paralelizáveis. Em adendo, as ANNs podem ser treinadas integrando o modelo matemático do problema diretamente na função de custo penalizando previsões que violam as relações físicas conhecidas. Essa metodologia, denominada Rede Neural Fenomenologicamente Informada (\textit{Physics-Informed Neural Network} — PINN), é especialmente vantajosa para simular sistemas complexos e com poucos dados experimentais por combinar a flexibilidade das redes neurais com o rigor das equações diferenciais do problema físico \citep{raissi_2019}.

Uma variação dessa abordagem são as Redes Neurais Recorrentes Fenomenologicamente Informadas (\textit{Physics-Informed Recurrent Neural Networks} — PIRNNs), que aprimoram a capacidade de capturar dependências temporais, sendo particularmente úteis para prever a dinâmica de sistemas que evoluem temporalmente (\citep{zheng_2023}).

Essa pesquisa tem como objetivo explorar o uso de PIRNNs como uma alternativa aos métodos convencionais na simulação de sistemas dinâmicos. Especificamente, visa-se avaliar a redução no tempo de cômputo e a qualidade das previsões em comparação com os métodos numéricos tradicionais. Além disso, investiga-se a aplicabilidade dessas redes em sistemas embarcados, demonstrando a viabilidade de embarcar uma PIRNN em um microcontrolador de baixo custo.
