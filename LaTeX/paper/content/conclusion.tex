\section{Considerações finais}

Neste trabalho, foi demonstrada a utilização de uma PIRNN embarcada em um Arduino UNO atuando como um analisador virtual para medição de nível de um conjunto de tanques.

Os resultados obtidos demonstram que a PIRNN é uma alternativa viável aos métodos numéricos tradicionais, apresentando desempenho superior no tempo de simulação, sem grandes prejuízos à fidelidade das previsões. Outro aspecto relevante é sua robustez, que se mantém mesmo na presença de falhas ou ruídos nas medições. Além disso, sua compatibilidade com plataformas de baixo custo, como o Arduino, torna-a uma solução acessível e eficiente para controle em tempo real, com aplicações potenciais em automação industrial, monitoramento de processos e como analisador virtual.

Contudo, em sistemas dinâmicos reais, é necessário considerar a degradação dos parâmetros do modelo ao longo do tempo, causada pelo desgaste ou envelhecimento dos componentes físicos. Essa degradação pode comprometer o desempenho da PIRNN, tornando necessário o retreinamento periódico do modelo para garantir sua acurácia. Nesse contexto, estudos futuros podem explorar técnicas de aprendizado por reforço, visando a adaptação em tempo real da PIRNN às mudanças dos parâmetros do sistema sem a necessidade de intervenções manuais.

Por fim, pesquisas adicionais podem investigar a implementação de modelos mais complexos, visando expandir a aplicabilidade das PIRNNs em sistemas dinâmicos descritos por equações diferenciais parciais.
