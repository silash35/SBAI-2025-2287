\section{Considerações finais}

\begin{frame}{Considerações finais}
  \begin{itemize}
    \item Usando o \textit{onnxruntime}, a PIRNN foi mais de três vezes mais rápida que os métodos numéricos tradicionais no mesmo computador.
    \item O Arduino demonstrou capacidade para executar PIRNNs, com ampla margem de recursos (\textbf{69,1\%} da memória flash e apenas \textbf{15,9\%} da memória RAM foram utilizados).
    \item Mesmo com ruído ou ausência de medições, a PIRNN manteve previsões de boa qualidade, demonstrando sua robustez.
    \item O controlador PI manteve o controle da planta recebendo apenas os valores previstos pela PIRNN, garantindo operação contínua mesmo sem medições diretas.
  \end{itemize}
\end{frame}
